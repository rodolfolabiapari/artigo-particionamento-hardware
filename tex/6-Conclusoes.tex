%!TEX root = ../main.tex
% !TeX encoding = UTF-8
\section{Conclusões}
   Lorem ipsum dolor sit amet, consectetur adipiscing elit. Suspendisse nisl augue, elementum ac urna ut, pretium interdum ipsum. Donec nunc tortor, sollicitudin ut dolor non, volutpat eleifend velit. Nam enim enim, lacinia sed ante eget, efficitur sodales eros. Suspendisse gravida venenatis convallis. Mauris non volutpat mauris. Nullam tempus, eros ac mollis tristique, eros mauris pellentesque odio, quis pellentesque purus lorem vel nisi. Etiam in ornare orci. Orci varius natoque penatibus et magnis dis parturient montes, nascetur ridiculus mus. Vestibulum malesuada eros felis, non mattis nunc porttitor vitae.
   
   \todo[inline]{TODO}

   Cras dictum consectetur dignissim. Curabitur porta quis est vitae luctus. Cras vitae ex faucibus, vestibulum sem mattis, varius mi. Duis cursus et elit id faucibus. Morbi tristique luctus condimentum. Curabitur quis ex sapien. In fermentum lorem nec sem viverra, ac rutrum lacus dignissim. Nam vestibulum erat id magna mollis aliquam. Nulla porttitor, eros sed scelerisque dapibus, diam ipsum congue neque, vitae malesuada augue leo tempus nisi. Ut eget gravida elit. Curabitur accumsan, ex eget mattis tempor, purus purus venenatis enim, rhoncus gravida lacus sem sed eros. Quisque finibus tellus tincidunt turpis bibendum tempus. Donec vestibulum magna lorem, vitae pellentesque justo condimentum et.

   % conference papers do not normally have an appendix
   \section*{Algoritmos Particionados}
    
    \begin{algorithm}[!ht] \footnotesize
        \KwIn{vetor buffer.}
        \KwOut{média, variância, desvio padrão.}
        
        \BlankLine
        \Begin{
            $\vars{soma}= \vars{variancia} = 0$;
            
            \For{$\func{length}(\vars{buffer})$}{$\vars{soma} \mathrel{+}= \vars{buffer}_i$;}
            
            $\vars{media} = \vars{soma} \div \func{length}(\vars{buffer})$\;
            \BlankLine
            
            
            \For{$\func{length}(\vars{buffer})$}{
                $\vars{variancia} \mathrel{+}= (\vars{buffer}_i - \vars{media})^2$\;
            }
            
            \BlankLine
            $\vars{variancia} \mathrel{\div}= \func{length}(\vars{buffer})$\;
            \BlankLine
            $\vars{dp} = \func{sqrt}(\vars{variancia})$\;
        }
        \caption{Método Estatístico.}
        \label{alg:statistic}
    \end{algorithm}
    
    
    \begin{algorithm}[!ht] \footnotesize
        \KwIn{vetor buffer, ponto.}
        \KwOut{distância interpolada.}
        
        \BlankLine
        \Begin{
            
            $\vars{nova\_distancia} = 0$\;
            
            \For{$\vars{i}\ in\ 1:\func{length}(\vars{buffer})$}{
                $\vars{c} = \vars{d} = 1$;
                
                \BlankLine
                \For{$\vars{j}\ in\ 1:\func{length}(\vars{buffer})$}{
                    \If{$\vars{i} \ne \vars{j}$}{
                        $\vars{c} \mathrel{\times}= \vars{ponto} - \vars{j}$;
                        \BlankLine
                        $\vars{d} \mathrel{\times}= \vars{i} - \vars{j}$;
                    }
                }
                \BlankLine
                $\vars{nova\_distancia} \mathrel{+}= \vars{buffer}_i \times \vars{c} \div \vars{d}$;
            }
            \BlankLine
            
            \lIf{$\vars{nova\_distancia} \ge 0$}{$\Return\ \vars{nova\_distancia}$}
            \lElse{$\Return\ 0$}
            
        }
        \caption{Método Interpolação por Lagrange.}
        \label{alg:lagrange}
    \end{algorithm}
    
    \begin{algorithm}[!ht] \footnotesize
        \KwIn{distância$_1$, distância$_2 = 0$, distância$_3 = 0$.}
        \KwOut{primaridade.}
        
        \BlankLine
        \Begin{
            
            $\vars{primo} = \vars{distancia}_1 + \vars{distancia}_2 + \vars{distancia}_3$;
            \BlankLine
            
            \lIf{$\vars{primo} \le 1$}{\Return 0}
            \BlankLine
            \lIf{$\vars{primo} = 2$}{\Return 1}
            \BlankLine
            
            \lIf{\vars{primo} $\func{mod}(2) = 0$}{$\vars{primo} \mathrel{+}= 1$}
            \BlankLine
            
            $\vars{divisor} = \vars{primo} \div 2$;
            
            \lIf{\vars{divisor} $\func{mod}(2) = 0$}{$\vars{divisor} \mathrel{+}= 1$}
            \BlankLine
            
            
            \While{\vars{divisor} $> 2$}{
                \lIf{\vars{primo} $\func{mod}(divisor) = 0$}{\Return 0}
                \BlankLine
                \vars{divisor} $\mathrel{-}= 2$;
            }
            \Return 1;
            
        }
        \caption{Método Número Primo.}
        \label{alg:prime}
    \end{algorithm}
    %
    \begin{algorithm}[!ht]
         \footnotesize
        \KwIn{distância lida; distância mínima.}
        \KwOut{boolean.}
        
        \BlankLine
        \Begin{
            \tcp{Se há risco, retorna TRUE}
            \lIf{\vars{distancia\_lida} $\le$ \vars{distancia\_minima}}{
                \Return 1
            }
            \lElse {
                \Return 0
            }
        }
        \caption{Método avaliação de Risco.}
        \label{alg:risk}
    \end{algorithm}


   % use section* for acknowledgment
   \section*{Agradecimentos}


   \todo{The authors would like to thank...}
   Agradecemos à Universidade Federal de Ouro Preto, ao CNPq e a FAPEMIG pelo subsídio dessa pesquisa.

   % trigger a \newpage just before the given reference
   % number - used to balance the columns on the last page
   % adjust value as needed - may need to be readjusted if
   % the document is modified later
   %\IEEEtriggeratref{8}
   % The "triggered" command can be changed if desired:
   %\IEEEtriggercmd{\enlargethispage{-5in}}
